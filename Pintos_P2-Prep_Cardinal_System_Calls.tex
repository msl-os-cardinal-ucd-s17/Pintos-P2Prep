\documentclass[11pt, letterpaper]{article}

\usepackage[top=1.5in, bottom=1in, left=1.25in, right=1.25in]{geometry}

\usepackage[dvipsnames]{xcolor}
\usepackage{listings}

\usepackage[title]{appendix}

\begin{document}
\lstset{language=C,breaklines,
                keywordstyle=\color{blue},
                stringstyle=\color{red},
                commentstyle=\color{ForestGreen}}

\title{Workshop 4 - System Calls}
\author{Team Cardinal\\CSCI 3453 Spring 2017}
\maketitle




\section{System Calls}

\subsection{Overview}
System calls are a consequence of an operating system built upon Dual-Mode Operation. The dual-mode paradigm allows the operating system to place instructions within two different execution contexts: "user mode" and "kernel mode". In user mode, an instruction must be checked if the process running it has the appropriate permissions. On the flip side, operations in kernel mode can execute without additional permissions. In other words, kernel mode allows unfettered access to actions which are otherwise restricted. Dual-mode operations force programmers to draw a line between what constitutes a protected kernel operation and what does not. For this to be possible, the operating system provides user processes with system calls to interface with the kernel. More specifically, user-mode processes use system calls to query the kernel to act as an intermediary. To aid in understanding the execution and control flow of system calls for our upcoming second Pintos project, this document provides a technical overview of user-kernel mode transitions and the internal interrupts used for system calls.

\subsection{Mode Transitions}

\subsubsection{User to Kernel}

\paragraph{System Calls:}The primary distinction between a system call and the other mode transitions is that a user process makes a voluntary request to switch modes whereas the other mode transitions are involuntary. In many cases, this request is a special instruction known as a \textbf{trap instruction} which changes the next instruction of the process to begin at a pre-defined block. Different trap instructions are used for each system call and thus behave similarly to function calls. However, note that the kernel is where the desired actions of the system call are executed. Upon completion, the kernel restores the program counter of the process to the instruction immediately following the trap and then restores any contextual data (such as register values).

\paragraph{Interrupts:}Interrupts that come from external sources cause a temporary transfer of control from the user mode to kernel 
mode. When a interrupt occurs then that event must be dealt with. The only way for this to occur is for the kernel to regain control 
of the processor and take action by running the appropriate interrupt handler. 
\paragraph{Processor Exceptions:} Processor exceptions are generated by hardware and are typically caused by a process trying to 
perform some type of undefined or restricted instruction such as dividing by zero or accessing memory outside of it's bounds. In 
these cases, the processes hands control to the kernel to run an exception handler which may or may not stop execution of the 
process depending on what the exception is.


\subsubsection{Kernel to User}
Kernel to User transfer is somewhat more straightforward. The kernel is tasked with starting new processes and after the kernel has 
completed its job, control is shifted back to user mode so the process can begin. After the kernel is finished handling an 
interrupt or system call, it restores the program counter of the interrupted process and switches back to user mode so that the process can resume.

\subsection{Internal Interrupts}
As opposed to external interrupts which are generated by events outside the CPU, internal interrupts are caused directly by CPU instructions. Furthermore, internal interrupts are \textbf{synchronous} (i.e. synchronized with CPU instructions) as opposed to external interrupts which are not synchronized with the CPU and can occur at any point during the execution of a instruction. System calls therefore manifest as internal interrupts because there exists a direct link between where the instruction is called in the program and when the interrupt occurs.

\subsection{System Calls in Pintos}
A major task in Pintos P2-User Programs will be implementing system calls to for process management and file system operations. These include functionality such as starting and exiting user processes, and operations for working with files such as reading/writing, creating and deleting files. A full list of the system calls that must be implemented can be found in section 3.3.4 of the Pintos Manual, or Appendix A of this document.

Some of the system calls directly access the file system. It is important to provide a synchronization mechanism that disallows race conditions on data from the file system. For example, in the write, read, and exec system calls, a global lock should be acquired that maintains single access to shared file system resources. This synchronization will allow any number of user programs to make system calls at once and consequently access the file system across threads. Synchronization of the system calls that access the file system will make the Pintos abstract file system thread-safe and "bullet-proof" to build upon for subsequent projects. 

\subsubsection{Assembly Instructions}
One question that remains is how to invoke the system call after it has been made from the user space. Pintos accomplishes this task 
in a way consistent with the simulated IA-32 architecture that it runs on. Pintos defines four macros in \textit{/lib/user/syscall.c}, one for 0, 1, 2, or 3 arguments that are passed to that particular system 
call. Each argument is pushed onto the stack before the actual system call is invoked by the 'int \$0x30' assembly instruction. Both the 
assembly instruction and the system call number that was passed into the function determine the correct position in an interrupt 
vector table. Each entry in the vector table points to the code needed to appropriately handle the event associated with the system
call number. An example of one system call function is shown below:
\begin{lstlisting}[frame=single,basicstyle=\footnotesize]
/* Invokes syscall NUMBER, passing argument ARG0, and returns the
   return value as an `int'. */
#define syscall1(NUMBER, ARG0)                                           \
        ({                                                               \
          int retval;                                                    \
          asm volatile                                                   \
            ("pushl %[arg0]; pushl %[number]; int $0x30; addl $8, %%esp" \
               : "=a" (retval)                                           \
               : [number] "i" (NUMBER),                                  \
                 [arg0] "g" (ARG0)                                       \
               : "memory");                                              \
          retval;                                                        \
        })
\end{lstlisting}

In this scenario, a user program might make a system call such as \textit{wait(pid)}. Because the wait system call takes only one 
argument, the aforementioned \textit{syscall1} would then be called using the parameters NUMBER which is the system call number 
corresponding to wait, and the pid which is the process id to wait on. Looking at the assembly code above, we can see that ARG0 
(the system call argument) and then NUMBER (the system call number) are pushed onto the stack before the 'int \$0x30' instruction 
is executed to direct the system to the correct code. System call numbers have already been defined in Pintos as an enumerated 
type and a complete list of system call numbers can be found in \textit{src/lib/syscall-nr.h}.


\section*{Open Questions}

\begin{enumerate}
\item What synchronization constructs should be used to track the state of a child process directly executed by its parent process? 
\item What synchronization constructs besides file system synchronization are required for the system calls "exec" and "wait"?
\item Assuming that the interrupt frame's stack pointer points to valid user-mode virtual memory, how can the system call handler verify that the memory of arbitrary user processes isn't unintentionally modified?
\item Abstracting the question posed in the Project 2 DESIGNDOC template, are there any possible advantages by changing the mapping between thread ID's and process ID's when dealing with system calls?
\item How do real-world operating systems (e.g. Linux) handle system calls with an arbitrary number of arguments? More specifically, is it possible (or worthwhile) to dynamically create system calls with an arbitrary number of arguments?
\end{enumerate} 



\section*{Conclusion}

Support for user programs is integral for the functionality of any operating system. As discussed, a protective mechanism must be in place for functionally sandboxed user-mode processes to communicate with a relatively privileged kernel. System calls act as the 
entry point for this communication and provide a number of benefits. By having a well defined set of system calls, kernel and user 
level code can be developed independently and still transfer between one another as long as they abide by the criteria set forth by 
the system call API. Furthermore, system calls act as a line of defense for the kernel. System calls define what actions user 
programs are allowed to request and they must be implemented in such a way that user programs cannot infiltrate the operation of the
kernel inappropriately. This will be a critical bit of functionality in Pintos P2 and foresight must be used so that Pintos is able
to handle a user program probing the kernel in any number of fringe cases.

\pagebreak
\begin{appendices}

\section{Pintos System Calls}
\begin{lstlisting}[frame=single,basicstyle=\footnotesize]
void halt (void) NO_RETURN;
	/* Terminates Pintos by calling power_off() */
	
void exit (int status) NO_RETURN;
	/* Terminates current user program */

pid_t exec (const char *file);
	/* Runs an executable file pointed at by the function argument */

int wait (pid_t);
	/* Cause the calling process to wait until the process given by pid_t dies */

bool create (const char *file, unsigned initial_size);
	/* Creates a new file called file with size initial_size */

bool remove (const char *file);
	/* Deletes file */

int open (const char *file);
	/* Opens file */

int filesize (int fd);
	/* Returns size of file in bytes */

int read (int fd, void *buffer, unsigned length);
	/* Read a specified number of bytes from file fd into buffer. */

int write (int fd, const void *buffer, unsigned length);
	/* Writes specified number of bytes from buffer into file fd */

void seek (int fd, unsigned position);
	/* Changes position for where the next byte is to be read from */

unsigned tell (int fd);
	/* Return the position of next byte to be read or written from file fd */

void close (int fd);
	/* Close file pointed at by fd */

\end{lstlisting}

\pagebreak

%\section{Other appendix here}
%\begin{lstlisting}[frame=single,basicstyle=\footnotesize]
%
%\end{lstlisting}

\end{appendices}

\pagebreak

\section*{References}
	
\begin{itemize}
\item Pintos Source Code
	\begin{itemize}
	\item lib/user/syscall.h
	\item src/lib/syscall-nr.h
	\end{itemize}
\item Operating Systems: Principles and Practices Second Edition by Thomas Anderson and Michael Dahlin
\item Pintos Manual by Ben Pfaff
\item IA-32 Intel Architecture Software Developer's Manual Volume 1: Basic Architecture
\end{itemize}


\end{document}